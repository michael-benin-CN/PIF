\documentclass{beamer}

\usepackage[ngerman]{babel}
\usepackage[utf8]{inputenc}

\usepackage{listings}
\setbeamercovered{transparent}

\usepackage[percent]{overpic}

\lstset{
  language=prolog,
  showstringspaces=false,
  aboveskip=-33pt
}

%\usetheme{Boadilla}
%\usetheme{Rochester}
%\usetheme{Rochester}
\usetheme[]{Goettingen}

%Kopf- und Fußzeile definieren
\setbeamertemplate{headline}
{%
\begin{overpic}[width=\paperwidth]
{kopf-hg.png}%
  \put(0,0){{~}{~}}%
\end{overpic}
}

\usecolortheme{dove} % :-)

\setbeamercovered{transparent}
\beamertemplatenavigationsymbolsempty
\setbeamertemplate{footline}[frame number]

%Titelseite
\title{Gestaltung von Prozessketten}
\author[A. Kazakova, B. Lüers]{Anastasia Kazakova, Bengt Lüers}

\institute[Universität Oldenburg]{
  \inst{}Fakultät 2 - Informatik, Wirtschafts- und Rechtswissenschaften}
  \titlegraphic{\includegraphics[scale=.4]{unisignet_r08_c2_cutted.png}
}

\date{\today}

\begin{document}

 \frame{\titlepage}

 \frame{\frametitle{Inhaltsverzeichnis}\tableofcontents[]}

 \section[Einleitung]{Einleitung}
 \begin{frame}
  \frametitle{Hintergründe}
\end{frame}

 \section[Prozessketten]{Prozessketten}
 \begin{frame}
  \frametitle{Hintergründe}
 \end{frame}

 \subsection[Spaghettidiagramme]{Spagehttidiagramme}
 \begin{frame}
  \frametitle{Hintergründe}

 \end{frame}

 \subsubsection[Syntax]{Syntax}
 \begin{frame}
  \frametitle{Hintergründe}

 \end{frame}

 \subsection[Beispiel IBM]{Beispiel IBM}
 \begin{frame}
  \frametitle{IBM-Kreditvergabe: Ausgangssituation}
  Prozess
  \begin{itemize}
    \item Aufträge werden über Tocherunternehmen finanziert
    \item 5 Mitarbeiter eingebunden:
    \begin{itemize}
      \item 1 Telefonistin nimmt Antrag auf
      \item 3 Experten prüfen und definieren Kondititionen
      \item 1 Schreibkraft verfasst Angebot
    \end{itemize}
  \end{itemize}
  Auswirkungen
  \begin{itemize}
    \item Intransparenz
    \item 6 Tage Durchlaufzeit
    \item Kunden lassen Aufträge platzen
  \end{itemize}
 \end{frame}

 \begin{frame}
  \frametitle{IBM-Kreditvergabe: Prozessanalyse}
  Prozessanalyse
  \begin{itemize}
    \item Vorgang muss nur 90 Minuten dauern
    \item Übrige Zeit wird durch Warten und Informationsweitergabe verbraucht
  \end{itemize}
 \end{frame}

 \begin{frame}
  \frametitle{IBM-Kreditvergabe: Endsituation}
  Prozess
  \begin{itemize}
    \item 4 Mitarbeiter eingebunden:
    \begin{itemize}
      \item 1 Case Manager übernimmt gesamten Ablauf
      \item Case Manager wird von Computersystem und
      \item 3 Experten unterstützt
    \end{itemize}
  \end{itemize}

  Auswirkungen
  \begin{itemize}
    \item Durchgehender Ansprechpartner
    \item 4 Stunden Durchlaufzeit
    \item Mitarbeiterzahl reduziert
    \item Durchsatz gesteigert
  \end{itemize}
 \end{frame}

 \begin{frame}
  \frametitle{IBM-Kreditvergabe: Visualisierung}
  $<$Bild: 4.5/$>$
 \end{frame}

 \section[Ereignisorientierte Prozessketten]{Ereignisorientierte Prozessketten}
 \begin{frame}
  \frametitle{Hintergründe}

 \end{frame}

 \subsection[Hintergründe]{Hintergründe}
 \begin{frame}
  \frametitle{Hintergründe}

 \end{frame}

\end{document}
