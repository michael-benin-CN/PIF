\documentclass{beamer}

\usepackage[ngerman]{babel}
\usepackage[utf8]{inputenc}

\usepackage{listings}
\setbeamercovered{transparent}

\usepackage[percent]{overpic}

\lstset{
  language=prolog,
  showstringspaces=false,
  aboveskip=-33pt
}

%\usetheme{Boadilla}
%\usetheme{Rochester}
%\usetheme{Rochester}
\usetheme[]{Goettingen}

%Kopf- und Fußzeile definieren
\setbeamertemplate{headline}
{%
\begin{overpic}[width=\paperwidth]
{kopf-hg.png}%
  \put(0,0){{~}{~}}%
\end{overpic}
}

\usecolortheme{dove} % :-)

\setbeamercovered{transparent}
\beamertemplatenavigationsymbolsempty
\setbeamertemplate{footline}[frame number]

%Titelseite
\title{Gestaltung von Prozessketten}
\author[A. Kazakova, B. Lüers]{Anastasia Kazakova, Bengt Lüers}

\institute[Universität Oldenburg]{
  \inst{}Fakultät 2 - Informatik, Wirtschafts- und Rechtswissenschaften}
  \titlegraphic{\includegraphics[scale=.4]{unisignet_r08_c2_cutted.png}
}

\date{\today}

\begin{document}

 \frame{\titlepage}

 \frame{\frametitle{Inhaltsverzeichnis}\tableofcontents[]}

 \section[Einleitung]{Einleitung}
 
 %schönes Zitat aus dem Buch
 \begin{frame}
 
 
 \textsc{\flqq If you can't describe what you are doing as a process, you don't know what you are doing.\frqq} 
 \\
 \medskip 
 \begin{flushright}
	 \begin{small}
		\emph{\textit{W. Edwards Deming, \\
 				Amerikanischer Unternehmensberater \\
				und Professor an der Columbia Universität}}
	\end{small}
 \end{flushright}
 
 \end{frame}
 
 \begin{frame}
  \frametitle{Prozess. Prozesstypen. Prozessstruktur}
  
  \textsc{Prozess} ist eine Folge von Aktivitäten zur Erstellung einer Leistung mit einem Anfang, einem Ende und einem Ziel.

\bigskip 
  \textsc{Prozesstypen}\\
  \begin{itemize}
  	\item Kernprozess
  	\item Steuerungsprozess
  	\item Unterstützungsprozesse
  \end{itemize}

\bigskip

\textsc{Prozessstruktur}
 
\end{frame}




 \section[Prozessketten]{Prozessketten}
 \begin{frame}
  \frametitle{Prozessstrukturanalyse}
  
  Prozessstrukturanalyse beschreibt, welche Aktivitäten in welcher Reihenfolge von wem durchgeführt werden. Durch diese Analyse können die Prozessabläufe gut visualisiert werden. \\
    \end{frame}

  \begin{frame}
    \frametitle{Grunde der Analyse}
 	 \begin{itemize}
  		\item um herauszufinden, ob die Prozesse optimal gestaltet sind 
  		\item um die Redundanzen und Irrläufer in der Struktur zu erkennen 
  		\item um die Optimierungsansätze sinnvoll anwenden zu können
  	\end{itemize}
  \end{frame}

 
 \begin{frame}
  \frametitle{Analysemethoden}
  
  \begin{itemize}
 	 \item Spaghetti-Diagramm
 	 \begin{itemize}
 	 	\item graphische Darstellung auf hohem Aggregationsniveau
 	 	\item identifiziert wesentliche Prozessschwachstellen
 	 	\item eignet sich für deren Optimierung
 	 \end{itemize}
 	 \bigskip
  	\item Ereignisorientierte Prozesskette
  	\begin{itemize}
 	 	\item stellt Details von Prozessen dar
 	 	\item Erkennt Detailprobleme
 	 	\item eignet sich fürs deren Beseitigen
 	 \end{itemize}
  	
  \end{itemize}
  
   
 \end{frame}

 \subsection[Spaghettidiagramme]{Spagehttidiagramme}
 \begin{frame}
  \frametitle{Hintergründe}

 \end{frame}

 \subsubsection[Syntax]{Syntax}
 \begin{frame}
  \frametitle{Hintergründe}

 \end{frame}

 \subsection[Beispiel IBM]{Beispiel IBM}
 \begin{frame}
  \frametitle{IBM-Kreditvergabe: Ausgangssituation}
  Prozess
  \begin{itemize}
    \item Aufträge werden über Tocherunternehmen finanziert
    \item 5 Mitarbeiter eingebunden:
    \begin{itemize}
      \item 1 Telefonistin nimmt Antrag auf
      \item 3 Experten prüfen und definieren Kondititionen
      \item 1 Schreibkraft verfasst Angebot
    \end{itemize}
  \end{itemize}
  Auswirkungen
  \begin{itemize}
    \item Intransparenz
    \item 6 Tage Durchlaufzeit
    \item Kunden lassen Aufträge platzen
  \end{itemize}
 \end{frame}

 \begin{frame}
  \frametitle{IBM-Kreditvergabe: Prozessanalyse}
  Prozessanalyse
  \begin{itemize}
    \item Vorgang muss nur 90 Minuten dauern
    \item Übrige Zeit wird durch Warten und Informationsweitergabe verbraucht
  \end{itemize}
 \end{frame}

 \begin{frame}
  \frametitle{IBM-Kreditvergabe: Endsituation}
  Prozess
  \begin{itemize}
    \item 4 Mitarbeiter eingebunden:
    \begin{itemize}
      \item 1 Case Manager übernimmt gesamten Ablauf
      \item Case Manager wird von Computersystem und
      \item 3 Experten unterstützt
    \end{itemize}
  \end{itemize}

  Auswirkungen
  \begin{itemize}
    \item Durchgehender Ansprechpartner
    \item 4 Stunden Durchlaufzeit
    \item Mitarbeiterzahl reduziert
    \item Durchsatz gesteigert
  \end{itemize}
 \end{frame}

 \begin{frame}
  \frametitle{IBM-Kreditvergabe: Visualisierung}
  $<$4.5/$>$
 \end{frame}

 \subsection{Optimierungsansätze}
 \begin{frame}
  \frametitle{Optimierungsansätze}
  Im Folgenden
  \begin{itemize}
    \item Zusammenfassungen bewährter Optimierungsansätze
  \end{itemize}
  Im Text
  \begin{itemize}
    \item Beispiele
  \end{itemize}
 \end{frame}

 \begin{frame}
  \frametitle{Entfall}
  $<$4.6.1$>$
  Trivialer Fall
  \begin{itemize}
    \item Idee: Weglassen von Aktivitäten kann mehr als den ausfallenden Gewinn sparen.
    \item Vorteil: Beschleunigung + Kostenersparnis
    \item Nachteil: Geringer Nutzen entfällt
  \end{itemize}
 \end{frame}

 \begin{frame}
  \frametitle{Beschleunigung}
  $<$4.6.2$>$
  klassische Optimierung
  \begin{itemize}
    \item Idee: Aktivitäten können durch Standardisierung beschleunigt werden.
    \item Vorteil: Qualitätssicherung
    \item Nachteil: Weniger Gestaltungspielraum
  \end{itemize}
 \end{frame}

 \begin{frame}
  \frametitle{Zusammenlegung}
  $<$4.6.3$>$
  Reibungsverluste
  \begin{itemize}
    \item Idee: Übergeben von Aufgaben kostet Zeit
    \item Vorteil: Übertragungsfehler werden vermieden
    \item Nachteil: Spezialisierungvorteile gehen verloren
  \end{itemize}
 \end{frame}

 \begin{frame}
  \frametitle{Automatisierung}
  $<$4.6.4$>$
  Langeweile
  \begin{itemize}
    \item Idee: Routineaufgaben können automatisiert werden
    \item Vorteil: mittelfristige Einsparungen
    \item Nachteil: Automatisierung erzeugt hohe Investitionen
  \end{itemize}
 \end{frame}

 \begin{frame}
  \frametitle{Verlagerung}
  $<$4.6.5$>$
  Outsourcing
  \begin{itemize}
    \item Idee: Interne Abläufe mit externen Abhängigkeiten können ausgelagert werden
    \item Vorteil: Abhängige können besser planen
    \item Nachteil: Erhöhte Banspruchung von Externen
  \end{itemize}
 \end{frame}

 \begin{frame}
  \frametitle{Reihenfolge}
  $<$4.6.6$>$
  Defragmentierung
  \begin{itemize}
    \item Idee: Nicht alle Aktivitäten in einer Sequenz müssen auf ein ander folgen
    \item Vorteil: Abhängige können besser planen
    \item Nachteil: Erhöhte Beanspruchung von Externen
  \end{itemize}
 \end{frame}

 \begin{frame}
  \frametitle{Parallelisierung}
  $<$4.6.7$>$
  Threading
  \begin{itemize}
    \item Idee: Einige Aktivitäten können nebenläufig passieren
    \item Vorteil: Massive Zeiteinsparungen möglich
    \item Nachteil: Höherer Koordinationsaufwand
  \end{itemize}
 \end{frame}

 \begin{frame}
  \frametitle{Vereinheitlichen der Verantwortung}
  $<$4.6.8$>$
  Idealzustand aus Sicht des Kunden
  \begin{itemize}
    \item Idee: \glqq One Face To The Customer\grqq
    \item Vorteil: Vermeiden von Übertragungsfehlern
    \item Nachteil: Erhöhter Koodinationsaufwand
  \end{itemize}
 \end{frame}

 \begin{frame}
  \frametitle{Arbeit in Interdisziplinären Teams}
  $<$4.6.9$>$
  Kooperation
  \begin{itemize}
    \item Idee: In einer Gruppe können Experten mehr als allein.
    \item Vorteil: Klare Aufgabe, Verantwortung und Anreize nötig.
    \item Nachteil: Effiziente Kommunikation
  \end{itemize}
 \end{frame}

 \begin{frame}
  \frametitle{Leistungsmessung}
  $<$4.6.10$>$
  Beeinflussen durch Beobachten
  \begin{itemize}
    \item Idee: Transparenz fördert Effizienz.
    \item Vorteil: Mitarbeiter optimierien ihre Aufgaben selbst
    \item Nachteil: Systematische Transparenzmachung muss durchgehalten werden.
  \end{itemize}
 \end{frame}

 \section[Ereignisorientierte Prozessketten]{Ereignisorientierte Prozessketten}
 \begin{frame}
  \frametitle{Hintergründe}

 \end{frame}

 \subsection[Hintergründe]{Hintergründe}
 \begin{frame}
  \frametitle{Hintergründe}

 \end{frame}

\end{document}
