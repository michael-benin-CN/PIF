\documentclass{beamer}

\usepackage[ngerman]{babel}
\usepackage[utf8]{inputenc}

\usepackage{listings}
\setbeamercovered{transparent}

\usepackage[percent]{overpic}

\lstset{
  language=prolog,
  showstringspaces=false,
  aboveskip=-33pt
}

%\usetheme{Boadilla}
%\usetheme{Rochester}
%\usetheme{Rochester}
\usetheme[]{Goettingen}

%Kopf- und Fußzeile definieren
\setbeamertemplate{headline}
{%
\begin{overpic}[width=\paperwidth]
{kopf-hg.png}%
  \put(0,0){{~}{~}}%
\end{overpic}
}

\usecolortheme{dove} % :-)

\setbeamercovered{transparent}
\beamertemplatenavigationsymbolsempty
\setbeamertemplate{footline}[frame number]

%Titelseite
\title{Gestaltung von Prozessketten}
\author[A. Kazakova, B. Lüers]{Anastasia Kazakova, Bengt Lüers}

\institute[Universität Oldenburg]{
  \inst{}Fakultät 2 - Informatik, Wirtschafts- und Rechtswissenschaften}
  \titlegraphic{\includegraphics[scale=.4]{unisignet_r08_c2_cutted.png}
}

\date{\today}

\begin{document}

 \frame{\titlepage}

 \frame{\frametitle{Inhaltsverzeichnis}\tableofcontents[]}

 \section[Einleitung]{Einleitung}
 
 %schönes Zitat aus dem Buch
 \begin{frame}
 
 
 \textsc{\flqq If you can't describe what you are doing as a process, you don't know what you are doing.\frqq} 
 \\
 \medskip 
 \begin{flushright}
	 \begin{small}
		\emph{\textit{W. Edwards Deming, \\
 				Amerikanischer Unternehmensberater \\
				und Professor an der Columbia Universität}}
	\end{small}
 \end{flushright}
 
 \end{frame}
 
 \begin{frame}
  \frametitle{Prozess. Prozesstypen. Prozessstruktur}
  
  \textsc{Prozess} ist eine Folge von Aktivitäten zur Erstellung einer Leistung mit einem Anfang, einem Ende und einem Ziel.

\bigskip 
  \textsc{Prozesstypen}\\
  \begin{itemize}
  	\item Kernprozess
  	\item Steuerungsprozess
  	\item Unterstützungsprozesse
  \end{itemize}

\bigskip

\textsc{Prozessstruktur}
 
\end{frame}




 \section[Prozessketten]{Prozessketten}
 \begin{frame}
  \frametitle{Prozessstrukturanalyse}
  
  Prozessstrukturanalyse beschreibt, welche Aktivitäten in welcher Reihenfolge von wem durchgeführt werden. Durch diese Analyse können die Prozessabläufe gut visualisiert werden. \\
    \end{frame}

  \begin{frame}
    \frametitle{Grunde der Analyse}
 	 \begin{itemize}
  		\item um herauszufinden, ob die Prozesse optimal gestaltet sind 
  		\item um die Redundanzen und Irrläufer in der Struktur zu erkennen 
  		\item um die Optimierungsansätze sinnvoll anwenden zu können
  	\end{itemize}
  \end{frame}

 
 \begin{frame}
  \frametitle{Analysemethoden}
  
  \begin{itemize}
 	 \item Spaghetti-Diagramm
  	\item Ereignisorientierte Prozesskette
  \end{itemize}
  
   
 \end{frame}

 \subsection[Spaghettidiagramme]{Spagehttidiagramme}
 \begin{frame}
  \frametitle{Hintergründe}

 \end{frame}

 \subsubsection[Syntax]{Syntax}
 \begin{frame}
  \frametitle{Hintergründe}

 \end{frame}

 \subsection[Beispiel IBM]{Beispiel IBM}
 \begin{frame}
  \frametitle{Hintergründe}

 \end{frame}

 \section[Ereignisorientierte Prozessketten]{Ereignisorientierte Prozessketten}
 \begin{frame}
  \frametitle{Hintergründe}

 \end{frame}

 \subsection[Hintergründe]{Hintergründe}
 \begin{frame}
  \frametitle{Hintergründe}

 \end{frame}

\end{document}
